%% 
%% Copyright 2019-2021 Elsevier Ltd
%% 
%% This file is part of the 'CAS Bundle'.
%% --------------------------------------
%% 
%% It may be distributed under the conditions of the LaTeX Project Public
%% License, either version 1.2 of this license or (at your option) any
%% later version.  The latest version of this license is in
%%    http://www.latex-project.org/lppl.txt
%% and version 1.2 or later is part of all distributions of LaTeX
%% version 1999/12/01 or later.
%% 
%% The list of all files belonging to the 'CAS Bundle' is
%% given in the file `manifest.txt'.
%% 
%% Template article for cas-dc documentclass for 
%% double column output.

\documentclass[a4paper,fleqn]{cas-dc}

% If the frontmatter runs over more than one page
% use the longmktitle option.

%\documentclass[a4paper,fleqn,longmktitle]{cas-dc}

% \usepackage[numbers]{natbib}
%\usepackage[authoryear]{natbib}
\usepackage[numbers,authoryear,longnamesfirst]{natbib}

%%%Author macros
\def\tsc#1{\csdef{#1}{\textsc{\lowercase{#1}}\xspace}}
\tsc{WGM}
\tsc{QE}
%%%

% Uncomment and use as if needed
%\newtheorem{theorem}{Theorem}
%\newtheorem{lemma}[theorem]{Lemma}
%\newdefinition{rmk}{Remark}
%\newproof{pf}{Proof}
%\newproof{pot}{Proof of Theorem \ref{thm}}

\begin{document}
\let\WriteBookmarks\relax
\def\floatpagepagefraction{1}
\def\textpagefraction{.001}

% Short title
\shorttitle{Unsupervised SAR Despeckling with Diffusion Model}    

% Short author
\shortauthors{S. Xiao, S. Zhang, Q. Xu et al. }  

% Main title of the paper
\title [mode = title]{Unsupervised SAR Despeckling with Diffusion Model}  

% Title footnote mark
% eg: \tnotemark[1]
% \tnotemark[<tnote number>] 

% Title footnote 1.
% eg: \tnotetext[1]{Title footnote text}
% \tnotetext[<tnote number>]{<tnote text>} 

% First author
%
% Options: Use if required
% eg: \author[1,3]{Author Name}[type=editor,
%       style=chinese,
%       auid=000,
%       bioid=1,
%       prefix=Sir,
%       orcid=0000-0000-0000-0000,
%       facebook=<facebook id>,Z
%       twitter=<twitter id>,
%       linkedin=<linkedin id>,
%       gplus=<gplus id>]

\author[1]{Author1}%!!!!!!!!!!

% \author[1]{Author1}[<options>]

% Corresponding author indication


% Footnote of the first author
\fnmark[1]

% Email id of the first author!!!!!!!!!!
\ead{abaaba@lalala.com}

% URL of the first author
% \ead[url]{<URL>}

% Credit authorship
% eg: \credit{Conceptualization of this study, Methodology, Software}!!!!!!!!!!
\credit{Conceptualization of this study, Methodology, Software, Writing}

% % Address/affiliation
% \affiliation[<aff no>]{organization={},
%             addressline={}, 
%             city={},
% %          citysep={}, % Uncomment if no comma needed between city and postcode
%             postcode={}, 
%             state={},
%             country={}}

% 第二位作者
\author[1]{Author2 \corref{cor1}}
% [<options>]

% Footnote of the second author
\fnmark[2]

% Email id of the second author
\ead{ }

% URL of the second author
% \ead[url]{}

% Credit authorship
\credit{supervision, writing}

% Address/affiliationthe Research Institute of Electronic Science and Technology, University of Electronic Science and Technology of China
\affiliation[1]{organization={University of Electronic Science and Technology of China},
            % addressline={}, 
            city={Chengdu},
%          citysep={}, % Uncomment if no comma needed between city and postcode
            postcode={611731}, 
            state={Sichuan},
            country={China}}

            
\affiliation[2]{organization={Chongqing Jiaotong University},
            % addressline={}, 
            city={Chongqing},
%          citysep={}, % Uncomment if no comma needed between city and postcode
            postcode={400074}, 
            state={Chongqing},
            country={China}}

            



% 第3位作者
\author[1]{Author3}%[单位号]{}
\fnmark[3]%角标号
\ead{ }%邮箱
\credit{Software, Writing}

% 第4位作者
\author[2{Kecheng Ge}%[单位号]{}
\fnmark[4]%角标号
\ead{ }%邮箱
\credit{Software, Writing}

% 第5位作者
\author[1]{Author4}%[单位号]{}
\fnmark[5]%角标号
\ead{ }%邮箱
\credit{Writing}

% 第6位作者
\author[1]{Author5}%[单位号]{}
\fnmark[6]%角标号
\ead{ }%邮箱
\credit{Writing}

% Corresponding author text
\cortext[cor1]{Corresponding author. the Research Institute of Electronic Science and Technology, University of Electronic Science and Technology of China}

% Footnote text!!!!!!!!!!
\fntext[1]{Author1 received the B.S. degree in school of information science and engineering from Chongqing Jiaotong University, Chonngqing, China, in 2022. He is pursuing the M.S. degree with the Research Institute of Electronic Science and Technology from University of Electronic Science and Technology of China, Chengdu, China. 
his research interests include machine learning, intelligent signal processing and synthetic aperture radar image processing.}

\fntext[2]{Author2 was born in Anhui, China, in 1980. He received the Ph.D. degree in signal and information processing from the Beijing Institute of Technology, Beijing, China, in 2007. In October 2007, he joined the Research Institute of Electronic Science and Technology, University of Electronic Science and Technology of China, Chengdu, China. In August 2009, he became an Associate Professor with the University of Electronic Science and Technology of China. From May 2014 to May 2015, he was a Visiting Scholar with the Department of Electrical and Computer Engineering, National University of Singapore. His major research interests include radar imaging (SAR/ISAR) and the application of frequency diverse array technology.}

% For a title note without a number/mark
%\nonumnote{}

% Here goes the abstract
\begin{abstract}
Synthetic aperture radar (SAR) despeckling is very important for practical applications. Since the deep learning based SAR despeckling models often rely heavily on the labeled training data and struggle to handle the noisy images with varying noise intensities, this paper proposes an unsupervised SAR despeckling method with the diffusion model, which consists of forward and reverse processes. In the forward process, Gaussian noise is gradually added to the clear optical image in the logarithmic domain until the image is heavily contaminated. Then, in the reverse process, the noise in the image is autoregressively predicted and removed through a U-net like neural network until the image is close to the clear image. Furthermore, this paper proposes a shifting and averaging based algorithm to process high resolution image in patches separately, which removes the model dependence on high video memory GPUs. Experiment results demonstrate that the proposed method can despeckle the SAR images with varying noise intensities effectively. Though the model’s training does not depend on clear SAR images, it has competitive performance to advanced supervised models. Additionally, it achieves better results compared with state-of-the-art unsupervised methods for real world SAR images.
\end{abstract}

% Use if graphical abstract is present
%\begin{graphicalabstract}
%\includegraphics{}
%\end{graphicalabstract}

% Research highlights
\begin{highlights}
\item 
\item 
\item 
\end{highlights}

% Keywords
% Each keyword is seperated by \sep
\begin{keywords}
 Synthetic Aperture Radar\sep Deep learning\sep Image despeckling\sep Unsupervised learning\sep Diffusion model.
\end{keywords}

\maketitle

% Main text
\section{}\label{}

% Numbered list
% Use the style of numbering in square brackets.
% If nothing is used, default style will be taken.
%\begin{enumerate}[a)]
%\item 
%\item 
%\item 
%\end{enumerate}  

% Unnumbered list
%\begin{itemize}
%\item 
%\item 
%\item 
%\end{itemize}  

% Description list
%\begin{description}
%\item[]
%\item[] 
%\item[] 
%\end{description}  

% Figure
% \begin{figure}[<options>]
% 	\centering
% 		\includegraphics[<options>]{}
% 	  \caption{}\label{fig1}
% \end{figure}


\begin{table}[<options>]
\caption{}\label{tbl1}
\begin{tabular*}{\tblwidth}{@{}LL@{}}
\toprule
  &  \\ % Table header row
\midrule
 & \\
 & \\
 & \\
 & \\
\bottomrule
\end{tabular*}
\end{table}

% Uncomment and use as the case may be
%\begin{theorem} 
%\end{theorem}

% Uncomment and use as the case may be
%\begin{lemma} 
%\end{lemma}

%% The Appendices part is started with the command \appendix;
%% appendix sections are then done as normal sections
%% \appendix

\section{}\label{}

% To print the credit authorship contribution details
\printcredits

%% Loading bibliography style file
%\bibliographystyle{model1-num-names}
\bibliographystyle{cas-model2-names}

% Loading bibliography database
\bibliography{}

% Biography
\bio{}
% Here goes the biography details.
\endbio

\bio{pic1}
% Here goes the biography details.
\endbio

\end{document}

